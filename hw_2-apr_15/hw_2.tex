\documentclass{article} % A4 paper and 11pt font size
\setcounter{secnumdepth}{0}

\usepackage{amssymb, amsmath, amsfonts}
\usepackage{moreverb}
\usepackage{graphicx}
\usepackage{enumerate}
\usepackage{graphics}
\usepackage[margin=1.25in]{geometry}
\usepackage{color}
\usepackage{tocloft}
\renewcommand{\cftsecleader}{\cftdotfill{\cftdotsep}}
\usepackage{array}
\usepackage{float}
\usepackage{hyperref}
\usepackage{textcomp}
\usepackage[makeroom]{cancel}
\usepackage{bbold}
\usepackage{alltt}
\usepackage{physics}
\usepackage{mathtools}
\usepackage[normalem]{ulem}
\usepackage{amsthm}
\usepackage{tikz}
\usetikzlibrary{positioning}
\usetikzlibrary{arrows}
\usepackage{pgfplots}
\usepackage{bigints}
\allowdisplaybreaks
\pgfplotsset{compat=1.12}

\theoremstyle{plain}
\newtheorem*{theorem*}{Theorem}
\newtheorem{theorem}{Theorem}
\newtheorem*{lemma*}{Lemma}
\newtheorem{lemma}{Lemma}

\newenvironment{definition}[1][Definition]{\begin{trivlist}
\item[\hskip \labelsep {\bfseries #1}]}{\end{trivlist}}

\newcommand{\E}{\varepsilon}
\def\Rl{\mathbb{R}}
\def\Cx{\mathbb{C}}

\newcommand{\Ei}{\text{Ei}}

\usepackage[T1]{fontenc} % Use 8-bit encoding that has 256 glyphs
\usepackage{fourier} % Use the Adobe Utopia font for the document - comment this line to return to the LaTeX default
\usepackage[english]{babel} % English language/hyphenation

\usepackage{sectsty} % Allows customizing section commands
\allsectionsfont{\centering \normalfont\scshape} % Make all sections centered, the default font and small caps

\usepackage{fancyhdr} % Custom headers and footers
\pagestyle{fancy} % Makes all pages in the document conform to the custom headers and footers
\fancyhead[L]{\bf Sam Fleischer}
\fancyhead[C]{\bf UC Davis \\ Applied Mathematics (MAT207C)} % No page header - if you want one, create it in the same way as the footers below
\fancyhead[R]{\bf Spring 2016}

\fancyfoot[L]{\bf } % Empty left footer
\fancyfoot[C]{\bf \thepage} % Empty center footer
\fancyfoot[R]{\bf } % Page numbering for right footer
\renewcommand{\headrulewidth}{0pt} % Remove header underlines
\renewcommand{\footrulewidth}{0pt} % Remove footer underlines
\setlength{\headheight}{25pt} % Customize the height of the header

\newcommand{\VEC}[2]{\left\langle #1, #2 \right\rangle}
\newcommand{\ran}{\text{\rm ran }}
\newcommand{\Hilb}{\mathcal{H}}

\DeclareMathOperator*{\esssup}{\text{ess~sup}}

\newcommand{\problem}[2]{
\vspace{.375cm}
\boxed{\begin{minipage}{\textwidth}
    \section{\bf #1}
    #2
\end{minipage}}
}

\numberwithin{equation}{section} % Number equations within sections (i.e. 1.1, 1.2, 2.1, 2.2 instead of 1, 2, 3, 4)
\numberwithin{figure}{section} % Number figures within sections (i.e. 1.1, 1.2, 2.1, 2.2 instead of 1, 2, 3, 4)
\numberwithin{table}{section} % Number tables within sections (i.e. 1.1, 1.2, 2.1, 2.2 instead of 1, 2, 3, 4)

\setlength\parindent{0pt} % Removes all indentation from paragraphs - comment this line for an assignment with lots of text

\newcommand{\horrule}[1]{\rule{\linewidth}{#1}} % Create horizontal rule command with 1 argument of height

\usepackage{xcolor}
\definecolor{light-gray}{gray}{0.9}

\title{ 
\normalfont \normalsize 
\textsc{UC Davis, Applied Mathematics (MAT207C), Spring 2016} \\ [25pt] % Your university, school and/or department name(s)
\horrule{2pt} \\[0.4cm] % Thin top horizontal rule
\Huge Homework \#2 \\ % The assignment title
\horrule{2pt} \\[0.5cm] % Thick bottom horizontal rule
}

\author{\huge Sam Fleischer} % Your name

\date{April 15, 2016} % Today's date or a custom date

\begin{document}\thispagestyle{empty}

\maketitle % Print the title

\makeatletter
\@starttoc{toc}
\makeatother

\pagebreak

%%%%%%%%%%%%%%%%%%%%%%%%%%%%%%%%%%%%%%
\problem{Problem 1}{Compute the swimming speed of an undulating sheet moving at zero Reynolds number between two walls on which the velocity is zero (in lab frame) located at $y = \pm L$ in the limit of low amplitude.  In the reference frame moving with the swimming, the shape of the swimmer is $y = A \sin(kx - \omega t)$.}






%%%%%%%%%%%%%%%%%%%%%%%%%%%%%%%%%%%%%%
\problem{Problem 2}{Suppose the position of a mass on a damped linear spring obeys the following equation $$m\ddot{x} + b\dot{x} + kx = 0,$$ where $m$, $b$, and $k$ are constants representing the mass, damping coefficient, and spring constant, respectively.
\begin{enumerate}[(a)]
    \item Each term in the above equation has dimensions of force.  Identify the dimensions of $b$ and $k$ in terms of mass, length, and time.
    \item Identify the three time scales in the problem and discuss their physical meaning.
    \item Present two different nondimensionalizations: one appropriate for the limit of vanishing friction and the other appropriate for the limit of vanishing mass.  Identify the small nondimensional parameter in each case.
\end{enumerate}}
\begin{enumerate}[(a)]
    \item
        Since the dimensions of force are $\dfrac{\text{mass}\cdot\text{length}}{\text{time}^2}$, the dimensions of $x$ are $\text{length}$, and the dimensions of $\dot{x}$ are $\dfrac{\text{length}}{\text{time}}$, then the dimensions of $b$ are $\dfrac{\text{mass}}{\text{time}}$ and the dimensions of $k$ are $\dfrac{\text{mass}}{\text{time}^2}$.
    \item
        Let $L(T) = \dfrac{x(t)}{X}$ and $T = \dfrac{t}{\tau}$.  Then
        \begin{align*}
            \dot{L} = \frac{\dd L}{\dd T} = \frac{\dd L}{\dd x} \frac{\dd x}{\dd t} \frac{\dd t}{\dd T} = \frac{\tau}{X}\dot{x} \qquad \text{and} \qquad \ddot{L} = \frac{\dd}{\dd T}\dot{L} = \frac{\dd}{\dd T} \qty[\frac{\tau}{X}\dot{x}] = \frac{\tau}{X} \frac{\dd}{\dd T} \dot{x} = \frac{\tau}{X}\frac{\dd t}{\dd T} \frac{\dd}{\dd t}\dot{x} = \frac{\tau^2}{X}\ddot{x}
        \end{align*}
        Thus,
        \begin{align*}
            m\ddot{L} + b\tau\dot{L} + k\tau^2 L = 0.
        \end{align*}
        There are three timescales:
        \begin{enumerate}[(i)]
            \item
                Let $\tau = \frac{m}{b}$.  Then
                \begin{align*}
                    \ddot{L} + \dot{L} + \E L = 0
                \end{align*}
                where $\E = \frac{km}{b^2}$.
            \item
                Let $\tau = \sqrt{\frac{m}{k}}$.  Then
                \begin{align*}
                    \ddot{L} + \E\dot{L} + L = 0
                \end{align*}
                where $\E = \frac{b}{\sqrt{km}}$.
            \item
                Let $\tau = \frac{b}{k}$.  Then
                \begin{align*}
                    \E\ddot{L} + \dot{L} + L = 0
                \end{align*}
                where $\E = \frac{km}{b^2}$.
        \end{enumerate}
    \item
        The nondimensionalization appropriate for the limit of vanishing friction is the second of the three given above:
        \begin{align*}
            \ddot{L} + \E\dot{L} + L = 0
        \end{align*}
        where $\E = \frac{b}{\sqrt{km}}$.
    \item
        The nondimensionalization appropriate for the limit of vanishing mass is the third of the three given above:
        \begin{align*}
            \E\ddot{L} + \dot{L} + L = 0
        \end{align*}
        where $\E = \frac{km}{b^2}$.
\end{enumerate}






\end{document}
