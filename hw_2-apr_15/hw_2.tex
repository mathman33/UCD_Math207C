\documentclass{article} % A4 paper and 11pt font size
\setcounter{secnumdepth}{0}

\usepackage{amssymb, amsmath, amsfonts}
\usepackage{moreverb}
\usepackage{graphicx}
\usepackage{enumerate}
\usepackage{graphics}
\usepackage[margin=1.25in]{geometry}
\usepackage{color}
\usepackage{tocloft}
\renewcommand{\cftsecleader}{\cftdotfill{\cftdotsep}}
\usepackage{array}
\usepackage{float}
\usepackage{hyperref}
\usepackage{textcomp}
\usepackage[makeroom]{cancel}
\usepackage{bbold}
\usepackage{alltt}
\usepackage{physics}
\usepackage{mathtools}
\usepackage[normalem]{ulem}
\usepackage{amsthm}
\usepackage{tikz}
\usetikzlibrary{positioning}
\usetikzlibrary{arrows}
\usepackage{pgfplots}
\usepackage{bigints}
\allowdisplaybreaks
\pgfplotsset{compat=1.12}

\theoremstyle{plain}
\newtheorem*{theorem*}{Theorem}
\newtheorem{theorem}{Theorem}
\newtheorem*{lemma*}{Lemma}
\newtheorem{lemma}{Lemma}

\newenvironment{definition}[1][Definition]{\begin{trivlist}
\item[\hskip \labelsep {\bfseries #1}]}{\end{trivlist}}

\newcommand{\E}{\varepsilon}
\def\Rl{\mathbb{R}}
\def\Cx{\mathbb{C}}

\newcommand{\Ei}{\text{Ei}}

\usepackage[T1]{fontenc} % Use 8-bit encoding that has 256 glyphs
\usepackage{fourier} % Use the Adobe Utopia font for the document - comment this line to return to the LaTeX default
\usepackage[english]{babel} % English language/hyphenation

\usepackage{sectsty} % Allows customizing section commands
\allsectionsfont{\centering \normalfont\scshape} % Make all sections centered, the default font and small caps

\usepackage{fancyhdr} % Custom headers and footers
\pagestyle{fancy} % Makes all pages in the document conform to the custom headers and footers
\fancyhead[L]{\bf Sam Fleischer}
\fancyhead[C]{\bf UC Davis \\ Applied Mathematics (MAT207C)} % No page header - if you want one, create it in the same way as the footers below
\fancyhead[R]{\bf Spring 2016}

\fancyfoot[L]{\bf } % Empty left footer
\fancyfoot[C]{\bf \thepage} % Empty center footer
\fancyfoot[R]{\bf } % Page numbering for right footer
\renewcommand{\headrulewidth}{0pt} % Remove header underlines
\renewcommand{\footrulewidth}{0pt} % Remove footer underlines
\setlength{\headheight}{25pt} % Customize the height of the header

\newcommand{\VEC}[2]{\left\langle #1, #2 \right\rangle}
\newcommand{\ran}{\text{\rm ran }}
\newcommand{\Hilb}{\mathcal{H}}
\newcommand{\lap}{\Delta}

\DeclareMathOperator*{\esssup}{\text{ess~sup}}

\newcommand{\problem}[2]{
\vspace{.375cm}
\boxed{\begin{minipage}{\textwidth}
    \section{\bf #1}
    #2
\end{minipage}}
}

\numberwithin{equation}{section} % Number equations within sections (i.e. 1.1, 1.2, 2.1, 2.2 instead of 1, 2, 3, 4)
\numberwithin{figure}{section} % Number figures within sections (i.e. 1.1, 1.2, 2.1, 2.2 instead of 1, 2, 3, 4)
\numberwithin{table}{section} % Number tables within sections (i.e. 1.1, 1.2, 2.1, 2.2 instead of 1, 2, 3, 4)

\setlength\parindent{0pt} % Removes all indentation from paragraphs - comment this line for an assignment with lots of text

\newcommand{\horrule}[1]{\rule{\linewidth}{#1}} % Create horizontal rule command with 1 argument of height

\usepackage{xcolor}
\definecolor{light-gray}{gray}{0.9}

\title{ 
\normalfont \normalsize 
\textsc{UC Davis, Applied Mathematics (MAT207C), Spring 2016} \\ [25pt] % Your university, school and/or department name(s)
\horrule{2pt} \\[0.4cm] % Thin top horizontal rule
\Huge Homework \#2 \\ % The assignment title
\horrule{2pt} \\[0.5cm] % Thick bottom horizontal rule
}

\author{\huge Sam Fleischer} % Your name

\date{April 15, 2016} % Today's date or a custom date

\begin{document}\thispagestyle{empty}

\maketitle % Print the title

\makeatletter
\@starttoc{toc}
\makeatother

\pagebreak

%%%%%%%%%%%%%%%%%%%%%%%%%%%%%%%%%%%%%%
\problem{Problem 1}{Compute the swimming speed of an undulating sheet moving at zero Reynolds number between two walls on which the velocity is zero (in lab frame) located at $y = \pm L$ in the limit of low amplitude.  In the reference frame moving with the swimming, the shape of the swimmer is $y = A \sin(kx - \omega t)$.}

Stokes Equations are
\begin{align*}
    \lap \underline{u} - \grad p &= 0 \\
    \div y &= 0
\end{align*}
Assume the height of the undulating sheet is given by $y = y(x,t)$
\begin{align*}
    y(x,t) = A\sin(kx - \omega t)
\end{align*}
Since the reference frame moves with the swimmer, the $x$ component of the velocity vector is $0$ at any given point, and the $y$ component of the velocity vector is $y_t(x,t)$, so the velocity vector $v$ is
\begin{align*}
    v(x, y(x,t)) = \qty[\begin{array}{c} 0 \\ -\omega A\cos(kx - \omega t) \end{array}]
\end{align*}
Since we assume the flow is two-dimensional and incompressible, then there is a stream function $\psi$ such that
\begin{align*}
    \underline{u} = \qty[\begin{array}{c} u \\ v\end{array}] = \qty[\begin{array}{c} \psi_y \\ -\psi_x \end{array}] = \qty[\begin{array}{c} 0 \\ -\omega A\cos(kx - \omega t) \end{array}]
\end{align*}
The two-dimensional Laplacian $\lap = \qty(\partial_{xx} + \partial_{yy})$ can then be applied:
\begin{align*}
    \qty(\partial_{xx} + \partial_{yy})\qty[\begin{array}{c} \psi_y \\ -\psi_x \end{array}] - \qty[\begin{array}{c} p_x \\ p_y \end{array}] &= \qty[\begin{array}{c} 0 \\ 0 \end{array}] \\
    \implies \qty[\begin{array}{cc}\psi_{yxx} + \psi_{yyy} \\ -\psi_{xxx} - \psi_{xyy}\end{array}] - \qty[\begin{array}{c} p_x \\ p_y \end{array}] &= \qty[\begin{array}{c} 0 \\ 0 \end{array}] \\
    \implies \begin{cases}
        \psi_{yxxy} + \psi_{yyyy} - p_{xy} &= 0 \\
        \psi_{xxxx} + \psi_{xyyx} + p_{yx} &= 0
    \end{cases} \\
    \implies \psi_{xxxx} + 2\psi_{xxyy} + \psi_{yyyy} &= 0 \\
    \implies \lap^2\psi = \qty(\partial_{xx} + \partial_{yy})^2\psi &= 0
\end{align*}
Next we express $\psi_x$ and $\psi_y$ as Taylor Series', taken as $A \rightarrow 0$:
\begin{align*}
    \omega A \cos(kx - \omega t) &= \psi_x\qty(x, A\sin(kx - \omega t)) \\
    &= \psi_x(x,0) + A\sin(kx - \omega t)\psi_{xy}(x,0) + \frac{A^2\sin^2\qty(kx - \omega t)}{2}\psi_{xyy}(x,0) + O(A^3) \\
    0 &= \psi_y\qty(x, A\sin(kx - \omega t)) \\
    &= \psi_y(x,0) + A\sin(kx - \omega t)\psi_{yy}(x,0) + \frac{A^2\sin^2\qty(kx - \omega t)}{2}\psi_{yyy}(x,0) + O(A^3)
\end{align*}
Next we assume $\psi$ can be expanded in the asymptotic basis $\{1,A,A^2,\dots\}$:
\begin{align*}
    \psi(x,y) = \psi_0(x,y) + A\psi_1(x,y) + A^2\psi_2(x,y) + A^3\psi_3(x,y) + O(A^4)
\end{align*}
Assuming the swimmer is moving at a constant speed $S$, we can express $S$ in the asymptotic basis $\{1, A, A^2, \dots\}$:
\begin{align*}
    S = s_0 + As_1 + A^2s_2 + \dots
\end{align*}
and this is the constant speed at which the walls move through the moving reference frame.  Thus the other boundary conditions of the PDE given above are
\begin{align*}
    \psi_x(x, \pm L) = 0, \qquad \text{and} \qquad \psi_y(x, \pm L) = S
\end{align*}
Thus the differential equation we are trying to solve, along with boundary conditions, is
\begin{align*}
    \left\{\begin{array}{rl}
        \lap^2 \psi &=\ \ \ \ 0 \\
        \psi_{x}\qty(x, A\sin(kx - \omega t)) &=\ \ \ \ \omega A\cos(kx - \omega t) \\
        \psi_{y}\qty(x, A\sin(kx - \omega t)) &=\ \ \ \ 0 \\
        \psi_{x}\qty(x, \pm L) &=\ \ \ \ 0 \\
        \psi_{y}\qty(x, \pm L) &=\ \ \ \ S
    \end{array}\right.
\end{align*}
Next we utilize the asymptotic expansion of $\psi$ (using the basis $\{1,A,A^2,\dots\}$) to solve for its first few components.  First, the $O(1)$ components:
\begin{align*}
    \left\{\begin{array}{rl}
        \lap^2 \psi_0 &=\ \ \ \ 0 \\
        \psi_{0_x}\qty(x, 0) &=\ \ \ \ 0 \\
        \psi_{0_y}\qty(x, 0) &=\ \ \ \ 0 \\
        \psi_{0_x}\qty(x, \pm L) &=\ \ \ \ 0 \\
        \psi_{0_y}\qty(x, \pm L) &=\ \ \ \ s_0
    \end{array}\right.
\end{align*}
This implies $\psi_1 \equiv 0$.  Next, the $O(A)$ components:
\begin{align*}
    \left\{\begin{array}{rl}
        \lap^2 \psi_1 &=\ \ \ \ 0 \\
        \psi_{1_x}\qty(x, 0) &=\ \ \ \ \omega \cos(kx - \omega t) \\
        \psi_{1_y}\qty(x, 0) &=\ \ \ \ 0 \\
        \psi_{1_x}\qty(x, \pm L) &=\ \ \ \ 0 \\
        \psi_{1_y}\qty(x, \pm L) &=\ \ \ \ s_1
    \end{array}\right.
\end{align*}
To solve this, assume the solution has the Fourier form
\begin{align*}
    \psi_1(x,y) = \sum_{\substack{n \in \mathbb{N} \\ n \neq 0}} \qty(\qty(A_1 + B_1 y)\sinh(nky) + \qty(C_1 + D_1 y)\cosh(nky))\sin(kx - \omega t) + E_1 + F_1 y + G_1 y^2 + H_1 y^3
\end{align*}
Because the flow must be bounded, $G_1 = H_1 = 0$.  Since all of the boundary conditions are derivatives, we can disregard the constant term $E_1$, i.e.~without loss of generality, we can assume $E_1 = 0$.  Also, since the only nonzero Fourier mode in the $x$-derivative has $n = 1$, then
\begin{align*}
    \psi_1(x,y) = \qty(\qty(A_1 + B_1 y)\sinh(ky) + \qty(C_1 + D_1 y)\cosh(ky))\sin(kx - \omega t) + F_1 y
\end{align*}
Now we consider the boundary conditions:
\begin{align*}
    \psi_{1_x}(x,y) &= k\qty(\qty(A_1 + B_1 y)\sinh(ky) + \qty(C_1 + D_1 y)\cosh(ky))\cos(kx - \omega t) \\
    \implies \omega\cos(kx - \omega t) = \psi_{1_x}(x,0) &= kC_1\cos(kx - \omega t) \\
    \implies C_1 &= \frac{\omega}{k} \\
    \implies \psi_1(x,y) &= \qty(\qty(A_1 + B_1 y)\sinh(ky) + \qty(\frac{\omega}{k} + D_1 y)\cosh(ky))\sin(kx - \omega t) + F_1 y
\end{align*}
Also,
\begin{align*}
    \psi_{1_y}(x,y) &= \qty(\qty(D_1 + k(A_1 + B_1y))\cosh(ky) + \qty(B_1 + \omega + kD_1y)\sinh(ky))\sin(kx - \omega t)+ F_1 \\
    \implies 0 = \psi_{1_y}(x,0) &= (D_1 + k A_1)\sin(kx - \omega t) + F_1 \\
    \implies F_1 = 0 \qquad &\text{and} \qquad D_1 + kA_1 = 0 \\
    \implies \psi_1(x,y) &= \qty(\qty(A_1 + B_1 y)\sinh(ky) + \qty(\frac{\omega}{k} - k A_1 y)\cosh(ky))\sin(kx - \omega t)
\end{align*}
Using Maple, we can use the two boundary conditions $\psi_{1_x}(x, \pm L) = 0$ to solve for $A_1$ and $B_1$:
\begin{align*}
    A_1 = 0 \qquad &\text{and} \qquad B_1 = -\frac{\omega\cosh(kL)}{kL\sinh(kL)} \\
    \implies \psi_1(x,y) &= \qty(-\frac{\omega\cosh(kL)}{kL\sinh(kL)}y\sinh(ky) + \frac{\omega}{k}\cosh(ky))\sin(kx - \omega t)
\end{align*}
We can then use the boundary conditions $\psi_{1_y}(x, \pm L) = s_1$ to solve for $s_1$:
\begin{align*}
    \psi_{1_y}(x,y) &= \qty(-\frac{\omega\cosh(kL)}{kL\sinh(kL)}\omega - \frac{\cosh(kL)}{L\sinh(kL)}\omega y \cosh(ky) + \omega \sinh(k y))\sin(kx - \omega t) \\
    \implies s_1 = \psi_{1_y}(x,L) &= -\frac{\omega(kL + \cosh(kL)\sinh(kL))\sin(kx - \omega t)}{kL \sinh(kL)} \\
    \text{and }\ s_1 = \psi_{1_y}(x,-L) &= \frac{\omega(kL + \cosh(kL)\sinh(kL))\sin(kx - \omega t)}{kL \sinh(kL)} = -s_1
\end{align*}
Thus $s_1 = 0$.  However, since $kL > 0$ and $\omega > 0$, this is an apparent contradiction.






%%%%%%%%%%%%%%%%%%%%%%%%%%%%%%%%%%%%%%
\problem{Problem 2}{Suppose the position of a mass on a damped linear spring obeys the following equation $$m\ddot{x} + b\dot{x} + kx = 0,$$ where $m$, $b$, and $k$ are constants representing the mass, damping coefficient, and spring constant, respectively.
\begin{enumerate}[(a)]
    \item Each term in the above equation has dimensions of force.  Identify the dimensions of $b$ and $k$ in terms of mass, length, and time.
    \item Identify the three time scales in the problem and discuss their physical meaning.
    \item Present two different nondimensionalizations: one appropriate for the limit of vanishing friction and the other appropriate for the limit of vanishing mass.  Identify the small nondimensional parameter in each case.
\end{enumerate}}
\begin{enumerate}[(a)]
    \item
        Since the dimensions of force are $\dfrac{\text{mass}\cdot\text{length}}{\text{time}^2}$, the dimensions of $x$ are $\text{length}$, and the dimensions of $\dot{x}$ are $\dfrac{\text{length}}{\text{time}}$, then the dimensions of $b$ are $\dfrac{\text{mass}}{\text{time}}$ and the dimensions of $k$ are $\dfrac{\text{mass}}{\text{time}^2}$.
    \item
        Let $L(T) = \dfrac{x(t)}{X}$ and $T = \dfrac{t}{\tau}$.  Then
        \begin{align*}
            \dot{L} = \frac{\dd L}{\dd T} = \frac{\dd L}{\dd x} \frac{\dd x}{\dd t} \frac{\dd t}{\dd T} = \frac{\tau}{X}\dot{x} \qquad \text{and} \qquad \ddot{L} = \frac{\dd}{\dd T}\dot{L} = \frac{\dd}{\dd T} \qty[\frac{\tau}{X}\dot{x}] = \frac{\tau}{X} \frac{\dd}{\dd T} \dot{x} = \frac{\tau}{X}\frac{\dd t}{\dd T} \frac{\dd}{\dd t}\dot{x} = \frac{\tau^2}{X}\ddot{x}
        \end{align*}
        Thus,
        \begin{align*}
            m\ddot{L} + b\tau\dot{L} + k\tau^2 L = 0.
        \end{align*}
        There are three timescales:
        \begin{enumerate}[(i)]
            \item
                Let $\tau = \frac{m}{b}$.  Then
                \begin{align*}
                    \ddot{L} + \dot{L} + \E L = 0
                \end{align*}
                where $\E = \frac{km}{b^2}$.  The solution to this differential equation is
                \begin{align*}
                    L(T) &= A\exp[\lambda_1 T] + B\exp[\lambda_2 T]
                \end{align*}
                where
                \begin{align*}
                    \lambda_1, \lambda_2 = \frac{1}{2}\qty[-1 \pm \sqrt{1 - 4\E}]
                \end{align*}
                $\E > 0$ implies $\lambda_2 < 0 < \lambda 1$.  As $\E \rightarrow 0$, both $\lambda_1, \lambda_2 \rightarrow 0$.  This time scale represents the time to decay when the spring constant is exceptionally small.
            \item
                Let $\tau = \sqrt{\frac{m}{k}}$.  Then
                \begin{align*}
                    \ddot{L} + \E\dot{L} + L = 0
                \end{align*}
                where $\E = \frac{b}{\sqrt{km}}$.  The solution to this differential equation is
                \begin{align*}
                    L(T) &= A\exp[\lambda_1 T] + B\exp[\lambda_2 T]
                \end{align*}
                where
                \begin{align*}
                    \lambda_1, \lambda_2 = \frac{1}{2}\qty[-\E \pm \sqrt{\E^2 - 4}]
                \end{align*}
                For $\E < 2$, we see both eigenvalues are imaginary, and as $\E \rightarrow 0$, both $\lambda_1,\lambda_2 \rightarrow \pm 2i$.  This timescale is the period of the oscillations in the absence of friction.
            \item
                Let $\tau = \frac{b}{k}$.  Then
                \begin{align*}
                    \E\ddot{L} + \dot{L} + L = 0
                \end{align*}
                where $\E = \frac{km}{b^2}$.  The solution to this differential equation is
                \begin{align*}
                    L(T) &= A\exp[\lambda_1 T] + B\exp[\lambda_2 T]
                \end{align*}
                where
                \begin{align*}
                    \lambda_1, \lambda_2 = \frac{1}{2\E}\qty[-1 \pm \sqrt{1 - 4\E}]
                \end{align*}
                When $\E < \frac{1}{4}$, $\lambda_1,\lambda_2 \in \Rl$.  In other words, when the mass is small enough (in comparison to spring constant and friction), there are no osciallations and only exponential decay.  This timescale is the time until the magnitude of the exponential decay overtakes the magnitude of the osciallations.
        \end{enumerate}
    \item
        The nondimensionalization appropriate for the limit of vanishing friction is the second of the three given above:
        \begin{align*}
            \ddot{L} + \E\dot{L} + L = 0
        \end{align*}
        where $\E = \frac{b}{\sqrt{km}}$.

        The nondimensionalization appropriate for the limit of vanishing mass is the third of the three given above:
        \begin{align*}
            \E\ddot{L} + \dot{L} + L = 0
        \end{align*}
        where $\E = \frac{km}{b^2}$.
\end{enumerate}






\end{document}
