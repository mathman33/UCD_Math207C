\documentclass{article} % A4 paper and 11pt font size
\setcounter{secnumdepth}{0}

\usepackage{amssymb, amsmath, amsfonts}
\usepackage{moreverb}
\usepackage{graphicx}
\usepackage{enumerate}
\usepackage{graphics}
\usepackage[margin=1.25in]{geometry}
\usepackage{color}
\usepackage{tocloft}
\renewcommand{\cftsecleader}{\cftdotfill{\cftdotsep}}
\usepackage{array}
\usepackage{float}
\usepackage{hyperref}
\usepackage{textcomp}
\usepackage[makeroom]{cancel}
\usepackage{bbold}
\usepackage{alltt}
\usepackage{physics}
\usepackage{mathtools}
\usepackage[normalem]{ulem}
\usepackage{amsthm}
\usepackage{tikz}
\usetikzlibrary{positioning}
\usetikzlibrary{arrows}
\usepackage{pgfplots}
\usepackage{bigints}
\allowdisplaybreaks
\pgfplotsset{compat=1.12}

\theoremstyle{plain}
\newtheorem*{theorem*}{Theorem}
\newtheorem{theorem}{Theorem}
\newtheorem*{lemma*}{Lemma}
\newtheorem{lemma}{Lemma}

\makeatletter
\newcommand{\BIGG}{\bBigg@{3}}
\newcommand{\vast}{\bBigg@{4}}
\newcommand{\Vast}{\bBigg@{5}}
\makeatother

\newenvironment{definition}[1][Definition]{\begin{trivlist}
\item[\hskip \labelsep {\bfseries #1}]}{\end{trivlist}}

\newcommand{\E}{\varepsilon}
\def\Rl{\mathbb{R}}
\def\Cx{\mathbb{C}}

\newcommand{\Ei}{\text{Ei}}

\usepackage[T1]{fontenc} % Use 8-bit encoding that has 256 glyphs
\usepackage{fourier} % Use the Adobe Utopia font for the document - comment this line to return to the LaTeX default
\usepackage[english]{babel} % English language/hyphenation

\usepackage{sectsty} % Allows customizing section commands
\allsectionsfont{\centering \normalfont\scshape} % Make all sections centered, the default font and small caps

\usepackage{fancyhdr} % Custom headers and footers
\pagestyle{fancy} % Makes all pages in the document conform to the custom headers and footers
\fancyhead[L]{\bf Sam Fleischer}
\fancyhead[C]{\bf UC Davis \\ Applied Mathematics (MAT207C)} % No page header - if you want one, create it in the same way as the footers below
\fancyhead[R]{\bf Spring 2016}

\fancyfoot[L]{\bf } % Empty left footer
\fancyfoot[C]{\bf \thepage} % Empty center footer
\fancyfoot[R]{\bf } % Page numbering for right footer
\renewcommand{\headrulewidth}{0pt} % Remove header underlines
\renewcommand{\footrulewidth}{0pt} % Remove footer underlines
\setlength{\headheight}{25pt} % Customize the height of the header

\newcommand{\VEC}[2]{\left\langle #1, #2 \right\rangle}
\newcommand{\ran}{\text{\rm ran }}
\newcommand{\Hilb}{\mathcal{H}}
\newcommand{\lap}{\Delta}

\newcommand{\littleo}[1]{\text{\scriptsize$\mathcal{O}$}\qty(#1)}

\DeclareMathOperator*{\esssup}{\text{ess~sup}}

\newcommand{\problem}[2]{
\vspace{.375cm}
\boxed{\begin{minipage}{\textwidth}
    \section{\bf #1}
    #2
\end{minipage}}
}

\numberwithin{equation}{section} % Number equations within sections (i.e. 1.1, 1.2, 2.1, 2.2 instead of 1, 2, 3, 4)
\numberwithin{figure}{section} % Number figures within sections (i.e. 1.1, 1.2, 2.1, 2.2 instead of 1, 2, 3, 4)
\numberwithin{table}{section} % Number tables within sections (i.e. 1.1, 1.2, 2.1, 2.2 instead of 1, 2, 3, 4)

\setlength\parindent{0pt} % Removes all indentation from paragraphs - comment this line for an assignment with lots of text

\newcommand{\horrule}[1]{\rule{\linewidth}{#1}} % Create horizontal rule command with 1 argument of height

\usepackage{xcolor}
\definecolor{light-gray}{gray}{0.9}

\title{ 
\normalfont \normalsize 
\textsc{UC Davis, Applied Mathematics (MAT207C), Spring 2016} \\ [25pt] % Your university, school and/or department name(s)
\horrule{2pt} \\[0.4cm] % Thin top horizontal rule
\Huge Homework \#5 \\ % The assignment title
\horrule{2pt} \\[0.5cm] % Thick bottom horizontal rule
}

\author{\huge Sam Fleischer} % Your name

\date{May 9, 2016} % Today's date or a custom date

\begin{document}\thispagestyle{empty}

\maketitle % Print the title

\makeatletter
\@starttoc{toc}
\makeatother

\pagebreak

%%%%%%%%%%%%%%%%%%%%%%%%%%%%%%%%%%%%%%
\problem{Problem 1}{For the Van del Pol equation
\begin{align*}
    \E \dot{u} &= v + u - \frac{u^3}{3} \\
    \dot{v} &= -u
\end{align*}
we found equations for the inner an outer solutions in class.  On the inner and outer layers, one of the two equations above is in pseudo-steady-state.  Identify the values of $u$ and $v$ where there are corner layers.  Find the equations for the corner layer and identify the thickness of the layer.  Use a balance argument which involves retaining the time derivatives of both state variables.}
\begin{proof}
    The following system is considered to be the ``outer'' solution since $\dot{u} = \order{\frac{1}{\E}}$ and $\dot{v} = \order{1}$, that is, $u$ changes very rapidly and $v$ changes at a ``normal'' rate.
    \begin{equation}
        \begin{aligned}
            \E \dot{u} &= v + u - \frac{u^3}{3} \\
            \dot{v} &= -u.
        \end{aligned}
        \tag{outer}
    \end{equation}
    We can rescale the system by setting $T = \E t$, $U(T) = u(t)$, and $V(T) = v(t)$.  This is considered to be the ``inner'' solution since $\dot{U} = \order{1}$ and $\dot{V} = \order{\E}$.
    \begin{equation}
        \begin{aligned}
            \dot{U} &= V + U - \frac{U^3}{3} \\
            \dot{V} &= -\E U.
        \end{aligned}
        \tag{inner}
    \end{equation}
    The $\order{1}$ equations of the outer problem are
    \begin{equation}
        \begin{aligned}
            0 &= v + u - \frac{u^3}{3} \\
            \dot{v} &= -u
        \end{aligned}
        \tag{outer $\order{1}$}
        \label{outer_o_1}
    \end{equation}
    and the $\order{1}$ equations of the inner problem are
    \begin{equation}
        \begin{aligned}
            \dot{U} &= V + U - \frac{U^3}{3} \\
            \dot{V} &= 0.
        \end{aligned}
        \tag{inner $\order{1}$}
        \label{inner_o_1}
    \end{equation}
    The slow manifold is $v = \frac{u^3}{3} - u$.  This is a positive cubic with zeros $u = (0,0), (\pm\sqrt{3}, 0)$.  There is no $\dot{u}$ term (and hence no explicit horizontal flow) in (\ref{outer_o_1}), but $\dot{v} = -u$ shows us flow is in the negative $v$ direction on the positive $u$ plane and in the positive $v$ direction on the negative $u$ plane.  The flow is toward the maximum, $\qty(-1, \frac{2}{3})$, and the mimimum $\qty(1, -\frac{2}{3})$, and away from the origin.  However the only equilibrium point is $(0,0)$, so we must look at the (\ref{inner_o_1}) system.

    The fast flow is purely in the $U$ directior since $\dot{V} = 0$.  Above the slow manifold there is leftward fast flow, and below the manifold is rightward fast flow.  Since no flow approaches the equilibrium, we are expecting a limit cycle.  There is an inner layer whenever $(u, v) = \qty(\pm 1, \mp \frac{2}{3})$.  In $u$ this is a large sudden shift is value and in derivative.  In $v$ there is little shift in value but large shift in derivative.  This is the corner.

    Next make the following variable shifts:
    \begin{align*}
        u = 1 + \E^\gamma \bar{u} \qquad v = -\frac{2}{3} + \E^\beta \bar{v} \qquad t = t_0 + \E^\alpha \bar{t}.
    \end{align*}
    Since $\bar{t}$ is arbitrary, set $t_0 = 0$.  Then
    \begin{align*}
        \E^{1 + \gamma - \alpha} \dot{\bar{u}} &= \qty(\E^\beta\bar{v} - \frac{2}{3}) + \qty(\E^\gamma\bar{u} + 1) - \frac{\qty(\E^\gamma\bar{u} + 1)^3}{3} = \E^\beta\bar{v} - \frac{\qty(\E^{3\gamma}\bar{u}^3 + 3\E^{2\gamma}\bar{u}^2)}{3} \\
        \E^{\beta - \alpha}\dot{\bar{v}} &= -(\E^\gamma\bar{u} + 1).
    \end{align*}
    The only reasonable option for $\alpha$, $\beta$, and $\gamma$ is setting $\alpha = \beta = \frac{2}{3}$ and $\gamma = \frac{1}{3}$, which yields
    \begin{align*}
        \dot{\bar{u}} &= \bar{v} -  \bar{u}^2 - \frac{1}{3}\E\bar{u}^3 \\
        \dot{\bar{v}} &= -1 - \E^{\frac{1}{3}} \bar{u},
    \end{align*}
    and thus the thickness of the layer is $\E^{\frac{2}{3}}$. \pagebreak
\end{proof}
    






%%%%%%%%%%%%%%%%%%%%%%%%%%%%%%%%%%%%%%
\problem{Problem 2}{The FitzHigh-Nagumo equations are a model system from electrophysiology to describe the cross membrane electrical potential in neurons:
\begin{align*}
    \E \dot{v} &= v(a - v)(v - 1) - w \\
    \dot{w} &= v - \gamma w,
\end{align*}
The variable $v$ is the voltage and $w$ is called a recovery variable.  Suppose that $\E > 0$ is small, $0 < a < 1$, and $\gamma$ is sufficiently large that $v = 0$, $w = 0$ is the only equilibrium.

The single equilibrium is a global attractor: all trajectories approach it.  For some initial conditions, the solution quickly approaches the rest state.  For others the voltage rapidly rises to a large value for some period of time, then overshoots the equilibrium value, and finally approaches the rest state.  This second response is called an action potential.

Analyze the inner and outer later structures of this system to characterize which initial conditions lead to an action potential, and describe the structure of the action potential using a layer analysis.}
\begin{proof}
    Set $g(v) = v(a - v)(v - 1)$.  Then the outer $\order{1}$ problem is
    \begin{align*}
        w &= g(v) \\
        \dot{w} &= v - \gamma w
    \end{align*}
    and the inner $\order{1}$ problem is
    \begin{align*}
        \dot{V} &= g(V) - W \\
        \dot{W} &= 0
    \end{align*}
    The slow manifold is a negative cubic with zeros $(0, 0)$, $(a, 0)$, and $(1, 0)$.  $g'(v) = 0$ gives $v_{1,2} = \frac{(a + 1) \pm \sqrt{(a + 1)^3 - 3a}}{3}$ where $v_1 < v_2$.  Define $w_{i} = g(v_i)$ for $i = 1, 2$.  Thus the minimum is $(v_1, w_1)$ and the maximum is $(v_2, w_2)$.  The line $w = \frac{1}{\gamma}v$ separates the slow flow along this manifold.  Since the line is steep enough, all points on the manifold with $v > 0$ are beneath the line $x = \frac{1}{\gamma}v$.  Flow underneath this line is upward, and flow above this line is downward, so all slow flow approaches the equilibrium $(0, 0)$ and the maximum $(v_2, w_2)$.  The slow manifold can be broken into three parts: $g_1$: the decreasing part from $-\infty$ to $v_1$, $g_2$: the increasing part from $v_1$ to $v_2$, and $g_3$: the decreasing part from $v_2$ to $\infty$.  Trajectories starting above the curve $g_2$ or above the horizontal lines protruding leftward from the minimum and rightward from the maximum never reach $g_3$.  They quickly approach $g_1$ and increase or decrease toward the original.  Trajectories starting below $g_2$ or below the horizontal lines protruding leftward from the minimum and rightward from the maximum always reach the $g_3$ manifold before quickly returning to the $g_1$ manifold.  We expect a corner layer in $v$ and $w$ at $(v_2, w_2)$.

    Next make the following variable shifts:
    \begin{align*}
        v = v_2 + \E^\phi \bar{v} \qquad w = w_2 + \E^\beta \bar{w} \qquad t = t_0 + \E^\alpha \bar{t}.
    \end{align*}
    Since $\bar{t}$ is arbitrary, set $t_0 = 0$.  Using Maple, with the above substitutions gives
    \begin{align*}
        \E^{1 + \phi - \alpha}\dot{\bar{v}} = \E^{3\phi}\bar{v}^3 - \E^{2\phi}\bar{v}^2\sqrt{a^2 - a + a} - \E^\beta \bar{w} \\
        \E^{\beta - \alpha}\dot{\bar{w}} = v_2 - \gamma w_2 + \E^\phi \bar{v} - \gamma \E^\beta \bar{w}.
    \end{align*}
    The only reasonable option for $\alpha$, $\beta$, and $\phi$ is setting $\alpha = \beta = \phi = 1$, which yields
    \begin{align*}
        \E\dot{\bar{v}} = \E^3\bar{v}^3 - \E^2\bar{v}^2\sqrt{a^2 - a + a} - \E\bar{w} \\
        \dot{\bar{w}} = v_2 - \gamma w_2 + \bar{v} - \gamma \bar{w}.
    \end{align*}
    and thus the thickness of the layer is $\E^1$.
\end{proof}
    






\end{document}
